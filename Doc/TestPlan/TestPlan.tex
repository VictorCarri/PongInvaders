\documentclass[12pt, titlepage]{article}

\usepackage{booktabs}
\usepackage{tabularx}
\usepackage{hyperref}
\hypersetup{
    colorlinks,
    citecolor=black,
    filecolor=black,
    linkcolor=red,
    urlcolor=blue
}
\usepackage[round]{natbib}

\title{SE 3XA3: Test Plan\\Title of Project}

\author{Team \ 10, Ben 10
		\\ Rehan Theiveehathasan, theivers
		\\ Puru Jetly, jetlyp
		\\ Karnvir Bining, biningk
}

\date{\today}

\input{../Comments}

\begin{document}

\maketitle

\pagenumbering{roman}
\tableofcontents
\listoftables
\listoffigures

\begin{table}[bp]
\caption{\bf Revision History}
\begin{tabularx}{\textwidth}{p{3cm}p{2cm}X}
\toprule {\bf Date} & {\bf Version} & {\bf Notes}\\
\midrule
Oct 31, 2016 & 1.0 &  Initial Draft\\
\bottomrule
\end{tabularx}
\end{table}

\newpage

\pagenumbering{arabic}

\section{General Information}

\subsection{Purpose}
The purpose of testing Pong Invaders is, at its core, to ensure that
the game meets all its requirements whether technical, functional, or
usablilityrelated.


\subsection{Scope}
The scope of the project is a two phase project, that our team
believes to be moderately complex. Phase one is Space Invaders, while phase two
in Pong. Once both are completed the games can be merged into Pong Invaders.
With our workflow being split into two phases, our team can test functionality
in parallel. This also allows our team to ensure working subsystems of Space
Invaders, and Pong prior to Pong Invaders being created. Once testing is
complete for the subsystems, all that would remain to be tested - once Pong
Invaders is created - are merge conflicts which would be trivial to test for by
unit testing.

\subsection{Acronyms, Abbreviations, and Symbols}
	
\begin{table}[hbp]
\caption{\textbf{Table of Abbreviations}} \label{Table}

\begin{tabularx}{\textwidth}{p{3cm}X}
\toprule
\textbf{Abbreviation} & \textbf{Definition} \\
\midrule
A.I & Artificial Intelligence, is the programmed behaviour of non-player
controlled entities in the game.\\
Eng & Engineering\\
ETC & Et cetera\\
Ex & Example\\
git & Git Lab\\
J  & Java \\
OS & Operating System\\
Repo & Repository\\
txt & Text File\\
Repo & Repository\\
\bottomrule
\end{tabularx}

\end{table}

\begin{table}[!htbp]
\caption{\textbf{Table of Definitions}} \label{Table}

\begin{tabularx}{\textwidth}{p{3cm}X}
\toprule
\textbf{Term} & \textbf{Definition}\\
\midrule
Ben Ten & Project Team\\
Client & The group that this product is made for\\
Customer/User & Any individual or group that utilizes the product after completion\\
Product & The game that is being developed\\
Project & Development of the product\\
\bottomrule
\end{tabularx}

\end{table}	

\subsection{Overview of Document}
The purpose of this document is to describe the overall test plan on
how testing for Pong Invaders is to be conducted. The following test plan
described in this document will provide what framework will be used to test
Pong Invaders, as well as the test cases that will be examined.

\section{Plan}
	
\subsection{Software Description}
Pong Invaders is written in Java, and is a game that takes mouse or keyboard as
inputs. For output the game displays the game state and graphics to the
display.The keyboard input will be used to move the ship in the game, navigate
through
menus, and fire bullets. The mouse input will be used as well if the player so
desires, as the controls in game can be rebounded to the mouse. The functions
ofthe %% typo here
software that are to be tested are the: window, collision, keyinput,
menu,%%need a space here
player, bullet and alien functions. The window function handles the window
creation when the game is started. Collision handles all detection of sprites
with one another. Keyinput handles whether a key has been pressed and what to
do when that key is pressed. Menu handles the main menu game state. Player
handles the ship functionality and how it moves. Bullet handles how the ship
fires
bullets and how they are drawn. Lastly the alien function handles how the
aliens are drawn on the screen and their A.I.
%% for the paragraph above, you need "The collision function", "the KeyInput function", etc.
%% instead of using them as proper nouns

\subsection{Test Team}
The test team for Pong Invaders will consist of the initial developers: Puru
Jetly, Rehan Theiveehathasan, and Karnvir Bining.


\subsection{Automated Testing Approach}
The automated testing approach that our team is adopting is JUnit. We will be
using JUnit due to how widely used of a framework it is, as well as how easy it
is to learn,%% don't need this comma
 and implement. We plan on using JUnit to automate testing for %% the collision, alian, and keyinput modules/functions
collision, alien, and keyinput to see if those functions output the correct
type and expected values to other functions.


\subsection{Testing Tools}
For testing tools we will be using JUnit

\subsection{Testing Schedule}
Testing will be done one week prior to Rev0 being completed.

\section{System Test Description}
	
\subsection{Tests for Functional Requirements}

\subsubsection{Area of Testing1}
		
\paragraph{Menu and Peripherals Test}

\begin{enumerate}

\item{Mouse Coordinate Test\\}

Type: Manual 
					
Initial State: Active Game State 
					
Input: Mouse Movement 
					
Output: Real-Time Mouse Coordinates 
					
How test will be performed: With the game in an active running state, the
program will be set to print the mouse coordinates whenever the mouse is moved.
The mouse will be moved to the boundaries of the screen and it will be checked
that the coordinates are equal to the screen length and width of the screen
whenrespectively checked. %% typo here
					
\item{Input Key Test\\}

Type: Manual 
					
Initial State: Active Game State 
					
Input: Keyboard inputs
					
Output: Playable character Movement
					
How test will be performed: The game will be put in an active state and it will
be checked that all pre-set keys are functioning and the correct output is
givenin game. For example if the left and right arrow keys are the key bindings
for
ship movement then the ship should move on screen when keyboard input is given.


\item{Menu Test\\}

Type: Manual
					
Initial State: Active Game State
					
Input: Keyboard and mouse inputs 
					
Output: Correct Menu option is followed 
					
How test will be performed: The game will be put in an active menu state and
themenu%% typo
screeen will be used to traverse different menu options. The Play now
option will be checked to start the game. Followed by the how to play option
which should display a how to play guide. Finally a re-binding option should be
given so that the player can re-bind input keys for personal preference during
play.


\item{Key-binding Test\\}

Type: Manual

Initial State: Active Game State 

Input: Keyboard inputs

Output: Playable character Movement

How test will be performed: The game will be put in an active state and the
userwill re-bind their keys to a more personal fit. The keys will then be
checked tofunction correctly based on binding.

\end{enumerate}

\subsubsection{Area of Testing2}

\paragraph{Ingame Value Test Cases}

\begin{enumerate}

\item{Method Check\\}

Type: Functional

Initial State: In-Active Game State 

Input: J-Unit test cases  %% "JUnit" not "J-Unit"

Output: Relevant return values for tested methods 

How test will be performed: The game will be left in an inactive state. Several J-unit tests will be set up to give major methods values and check for the correct return values.

\item{Collision Tests\\}

Type: Dynamic

Initial State: Active Game State 

Input: none

Output: Collision identification dialogue box or print statement 

How test will be performed: The game will be in an active state and the player wil be set tol %% typo
simply shoot bullets and stay in a static position. The program will be set to print each time a successful collision is made. After 3 successful bullet collisions, the program will stop bullet shots and wait for the enemies to hit the playable ship to check for final collision. 

\item{Exit Game Test\\}

Type: Manual

Initial State: Active Game State 

Input: Keyboard and mouse input

Output: Game exits 

How test will be performed: After entering a running game, the user will exit using the associated key binding or by exiting the window entirely. In the first case, the user should return to the title screen with saved settings intact. In the second case, the user should be able to turn the game back on with saved settings intact. 

\item{Game Over Test\\}

Type: Static

Initial State: Active Game State 

Input: none

Output: Game Over Screen 

How test will be performed: Game start will be initiated and aliens will simply run down the screen until user character is hit. The collision should be followed shortly by a game over screen. 

\item{Invalid Key Test\\}

Type: manual

Initial State: Active Game State 

Input: Keyboard input

Output: Error messages

How test will be performed: Invalid keyboard inputs, unbound keys, will be input while game is in progress. If an invalid key is pressed twice or more a small dialogue box will appear remind the user that the key is unbound. 

\item{Sprite Test\\}

Type: Functional 

Initial State: Inactive Game State

Input: none

Output: Correct object sprite

How test will be performed: The program will be set to display each sprite when called from a j-unit test. Each sprite should match the true case in the j-unit test. 

\end{enumerate}

\subsection{Tests for Nonfunctional Requirements}

\subsubsection{Area of Testing1}
		
\paragraph{Stylistic tests}

\begin{enumerate}

\item{Re-size Test\\}

Type: Manual
					
Initial State: Active Game State
		
Input/Condition: Mouse input screen resizing
					
Output/Result: Screen size change
					
How test will be performed: The mouse will be used to make the screen bigger, this should happen while keeping the resolution of all objects the same. 
					
\item{Input Test\\}

Type: Manual
					
Initial State: Active Game State
					
Input: Mouse Movements and keyboard input
					
Output: real-time input tracking 
					
How test will be performed: multiple user inputs will be given, the system will have to print back all given inputs in real-time. 

\end{enumerate}

\subsubsection{Area of Testing2}

\paragraph{Maintainability and support tests}

\begin{enumerate}

\item{Legal Test\\}

Type: Manual
					
Initial State: Active Game State
					
Input: Mouse Movements and keyboard input
					
Output: none
					
How test will be performed: Game will be checked for any legal issues, no offensive language or icons 

\item{Development Test\\}

Type: Manual
					
Initial State: Inactive game state
					
Input: none
					
Output: none
					
How test will be performed: The game jar will simply be downloaded on a none master branch computer and checked to see if local development is possible. To allow for user development and personal use. 


\end{enumerate}


\section{Tests for Proof of Concept}

\subsection{Area of Testing1}
		
\paragraph{Space Invaders Testing}

\begin{enumerate}

\item{Movement of ship\\}

Type: Functional

Initial State: Application is running

Input/Condition: Keyboard buttons ‘a’ and ‘d'

Output/Result: Ship moves

How test will be performed: Test will be done manually by developers running prototype. 

					
\item{Hitbox of alien\\}

Type: Functional

Initial State: Application is running

Input/Condition: Keyboard ‘spacebar’

Output/Result: Ship shoots an alien. Alien dies.

How test will be performed: Test will be done manually by developers running prototype. Alien will be lined up with ship. Ship will then shoot a bullet towards to alien to check for correctness of hitboxes on alien.



\item{Hitbox of ship and end game screen\\}

Type: Functional

Initial State: Application is running

Input/Condition: Alien is on course to hit ship.

Output/Result: Ship takes damage. Player loses game

How test will be performed: Test will be done manually by developers running prototype. Ship will not move when game starts and Alien will hit the ship. Test done to see if end game screen can be reached by destroying the ship. 

\end{enumerate}

\subsection{Area of Testing2}

\paragraph{Pong Testing}

\begin{enumerate}

\item{Hitbox of paddle\\}

Type: Functional

Initial State: Application is running

Input/Condition: Pong paddle moves with ‘w’ and ‘s’ and can hit pong ball

Output/Result: Pong paddle moves up and down on left side of screen and hits ball back

How test will be performed: Test will be done manually by developers running prototype. Ball is set on course to hit paddle to check if physics on paddle are working.


\item{Hitbox of the alien\\}

Type: Functional

Initial State: Application is running

Input/Condition: Pong paddle hits pong ball towards an alien.

Output/Result: Alien blows up.

How test will be performed: Test will be done manually by developers running prototype. Ball is set on course to hit paddle then it rebounds to an alien. This test done to check if hit boxes on alien interact with pong ball.


\item{Point Scoring\\}

Type: Structural

Initial State: Application is running

Input/Condition: User controlled Pong paddle misses ball. 

Output/Result: Points taken off from user’s overall score.

How test will be performed: Test will be done manually by developers running prototype. Test is done to see if scoring of points is done correctly.

\end{enumerate}
	
\section{Comparison to Existing Implementation}	
The implementation of our game is somewhat different than the original. However, all requirements of the original have stayed the same in this implementation. 
One difference from the existing implementation is that there is a Pong overlay. Now the user must destroy the aliens coming for the ship while also playing a 
game of Pong at the same time. The Pong ball can interact with the aliens, which means that the ball can destroy aliens while also bouncing back to the original 
spot it came from. Another difference from the existing implementation is that the resolution of the game has increased from 800x400 to 1280x720. 
This way graphics can be presented in higher definition. Also the smoothness of the game has been increased by increasing the frames per second from 30 to 60.
				
\section{Unit Testing Plan}
		
\subsection{Unit testing of internal functions}

\begin{enumerate}

\item{Case 1\\}

Type: Functional

Initial State: Ships coordinate in x and y selected.

Input/Condition: Ship object.

Output/Result: Checks to see if Ships in x and y coordinates are drawn correctly to window.

How test will be performed: Ships coordinates in x and y will be selected before start of program.

\item{Case 2\\}

Type: Functional

Initial State: Aliens coordinate in x and y selected.

Input/Condition: Alien object.

Output/Result: Checks to see if aliens in x and y coordinates are drawn and are equal distant apart and drawn to window.

How test will be performed: Ships coordinates in x and y will be selected before start of program.

\item{Case 3\\}

Type: Functional

Initial State: Pong Paddles and Pong ball coordinate in x and y selected.

Input/Condition: Paddle and ball object.

Output/Result: Checks to see if paddles in x and y coordinates are drawn and are equal distant apart, also checking if ball spawns at correct spot and drawn to window.

How test will be performed: Paddles and ball coordinates in x and y will be selected before start of program.

\item{Case 4\\}

Type: Dynamic

Initial State: Bullet shot from ship.

Input/Condition: User shoots bullet.

Output/Result: Bullet does not change its x coordinate, y coordinate increases and then bullet is out of view. Bullet is then undrawn from to save computation.

How test will be performed: The user will shoot a bullet from ship

\item{Case 5\\}

Type: Dynamic

Initial State: Application is started.

Input/Condition: No input.

Output/Result: Fps counter on left side of screen shows if program running at 60fps.

How test will be performed: Application just needs to be run.

\item{Case 6\\}

Type: Dynamic

Initial State: Ship is drawn to correct x and y position.

Input/Condition: User tries to move ship with keys ‘a’ and ‘d’ and shoots bullets with ‘spacebar’

Output/Result: Ship moves to desired position of left or right. Ship shoots bullets when desired by user. 

How test will be performed: The user will move ship with keyboard and shoot with keyboard.

\item{Case 7\\}

Type: Dynamic

Initial State: User pong paddle is drawn to correct x and y position.

Input/Condition: User tries to move paddle with keys ‘w’ and ‘s’ 

Output/Result: User’s paddle moves up and down when desired by user.

How test will be performed: The user will move paddle with keyboard.

\item{Case 8\\}

Type: Functional

Initial State: No game objects in handler class.

Input/Condition: Program attempts to add a game object in linked list.

Output/Result: Game Object is added correctly to linked list.

How test will be performed: Elements in linked list printed to console.

\item{Case 9\\}

Type: Functional

Initial State: 2 game objects in handler class.

Input/Condition: Program attempts to delete a game object in linked list.

Output/Result: Game Object is deleted correctly from linked list.

How test will be performed: Elements in linked list printed to console.

\end{enumerate}
		
\subsection{Unit testing of output files}		

N/A

\bibliographystyle{plainnat}

\bibliography{SRS}

\newpage

\section{Appendix}

This is where you can place additional information.

\subsection{Symbolic Parameters}

The definition of the test cases will call for SYMBOLIC\_CONSTANTS.
Their values are defined in this section for easy maintenance.

\subsection{Usability Survey Questions?}

This is a section that would be appropriate for some teams.

\end{document}